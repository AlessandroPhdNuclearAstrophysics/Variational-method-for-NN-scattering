\documentclass[10pt,a4paper]{article}
\usepackage[latin1]{inputenc}
\usepackage{amsmath}
\usepackage{amsfonts}
\usepackage{amssymb}
\usepackage{graphicx}
\usepackage{hyperref}
\usepackage{amsmath, amsfonts,amsthm,bm,amssymb, graphics, setspace}
\usepackage{mathtools}
\usepackage[left=2.00cm, right=2.00cm, top=3.00cm, bottom=3.00cm]{geometry}
\usepackage{xfrac}
\usepackage{empheq}
\usepackage[most]{tcolorbox}
\usepackage{appendix}
\usepackage[table]{xcolor} 
\author{Alessandro Grassi}


\newtcbox{\mymathUtil}[1][]{%
	nobeforeafter, math upper, tcbox raise base,
	enhanced, colframe=blue!30!black,
	colback=white!10, boxrule=1pt,
	#1}

\newtcbox{\mymath}[1][]{%
	nobeforeafter, math upper, tcbox raise base,
	enhanced, colframe=blue!30!black,
	colback=blue!10, boxrule=1pt,
	#1}

\newtcbox{\mymathUsed}[1][]{%
	nobeforeafter, math upper, tcbox raise base,
	enhanced, colframe=blue!30!black,
	colback=yellow!30, boxrule=1pt,
	#1}

\DeclarePairedDelimiter\Bra{\langle}{\rvert}
\DeclarePairedDelimiter\Ket{\lvert}{\rangle}
\DeclarePairedDelimiterX\Braket[2]{\langle}{\rangle}{#1 \delimsize\vert #2}

\DeclarePairedDelimiter\wBra{\bigg\langle}{\bigg\rvert}
\DeclarePairedDelimiter\wKet{\bigg\lvert}{\bigg\rangle}
\DeclarePairedDelimiterX\wBraket[2]{\bigg\langle}{\bigg\rangle}{#1 \delimsize\vert #2}

\newcommand{\bra}[1]{\Bra{#1}}
\newcommand{\ket}[1]{\Ket{#1}}
\newcommand{\braket}[2]{\Braket{#1}{#2}}

\newcommand{\wbra}[1]{\wBra{#1}}
\newcommand{\wket}[1]{\wKet{#1}}
\newcommand{\wbraket}[2]{\wBraket{#1}{#2}}


\newcommand{\matrixelement}[3]{\bra{#1}\,#2\,\ket{#3}}
\newcommand{\wmatrixelement}[3]{\wbra{#1}\,#2\,\wket{#3}}
\newcommand{\matrixelementred}[3]{\langle #1 \Vert #2 \Vert #3\rangle}

\newcommand{\CG}[3]{\braket{#1\,#2\,}{#3}}
\newcommand{\eq}[1]{Eq.~#1}
\newcommand{\de}[1]{\mathrm{d}#1\,}
\newcommand{\ve}[1]{\mathbf{#1}}
\newcommand{\onehalf}{\sfrac{1}{2}\,}
\newcommand{\ylm}[3]{Y_{#1}^{#2}(#3)}
\newcommand{\versor}[1]{\ve{\hat{#1}}}
\newcommand{\mand}{\qquad\mathrm{and}\qquad}
\newcommand{\pd}[2]{\frac{\partial^{#1}}{\partial{#2}^{#1}}}
\newcommand{\mtm}[1]{\mathrm{#1}}



\newcommand{\threejsymbol}[6]{
	\begin{pmatrix}
		#1 & #2 & #3\\
		#4 & #5 & #6
	\end{pmatrix}
	}
\newcommand{\sixjsymbol}[6]{
	\begin{Bmatrix}
		#1 & #2 & #3\\
		#4 & #5 & #6
	\end{Bmatrix}
}

\mathtoolsset{showonlyrefs,showmanualtags}

\title{About indexes and matrix elements}


\begin{document}
	\maketitle
	\begin{table}[b]
		\centering
		\renewcommand{\arraystretch}{1.3}
		\setlength{\tabcolsep}{6pt}
		\begin{tabular}{|ccc|ccc|}
			\hline
			\textcolor{red}{(1,1)} & \textcolor{red}{(1,2)} & \textcolor{red}{(1,3)} & \textcolor{blue}{(1,4)} & \textcolor{blue}{(1,5)} & \textcolor{blue}{(1,6)} \\
			\textcolor{red}{(2,1)} & \textcolor{red}{(2,2)} & \textcolor{red}{(2,3)} & \textcolor{blue}{(2,4)} & \textcolor{blue}{(2,5)} & \textcolor{blue}{(2,6)} \\
			\textcolor{red}{(3,1)} & \textcolor{red}{(3,2)} & \textcolor{red}{(3,3)} & \textcolor{blue}{(3,4)} & \textcolor{blue}{(3,5)} & \textcolor{blue}{(3,6)} \\\hline
			\textcolor{green!60!black}{(4,1)} & \textcolor{green!60!black}{(4,2)} & \textcolor{green!60!black}{(4,3)} & \textcolor{cyan!80!black}{(4,4)} & \textcolor{cyan!80!black}{(4,5)} & \textcolor{cyan!80!black}{(4,6)} \\
			\textcolor{green!60!black}{(5,1)} & \textcolor{green!60!black}{(5,2)} & \textcolor{green!60!black}{(5,3)} & \textcolor{cyan!80!black}{(5,4)} & \textcolor{cyan!80!black}{(5,5)} & \textcolor{cyan!80!black}{(5,6)} \\
			\textcolor{green!60!black}{(6,1)} & \textcolor{green!60!black}{(6,2)} & \textcolor{green!60!black}{(6,3)} & \textcolor{cyan!80!black}{(6,4)} & \textcolor{cyan!80!black}{(6,5)} & \textcolor{cyan!80!black}{(6,6)} \\
			\hline
		\end{tabular}
		\caption{
			Example of a block structure of the $(NNL \times NCH) \times (NNL \times NCH)$ matrix for $NCH=2$, $NNL=3$.
			Each entry is of the form $(\mathrm{index}_\mathrm{row}, \mathrm{index}_\mathrm{col})$.
			Text color highlights the four $3\times3$ channel ($\alpha$, $\alpha'$) blocks: (1,1) in red, (1,2) in blue, (2,1) in green, (2,2) in cyan.
		}
		\label{tab:example}
	\end{table}
	
	Since we have two indexes in our calculations $\alpha$ and $n$, $\alpha$ being the scattering channel quantum numbers and $n$ being the index $n$ of the modified Laguerre base function $f_n^{(2)}(r)$, we flatten these two indexes into one common index
	\begin{verbatim}
		INDEX = 0
		DO I=1, NEQ
			DO J=1, NNL
				INDEX = INDEX + 1
				COMMON_INDEX(I, J) = INDEX
			ENDDO
		ENDDO
	\end{verbatim}
	Thus flattening the two indexes into one.
	From
	\begin{center}
		\begin{tabular}{c|ccc}	
			 & 1 & 2 & 3 \\\hline
			 1 & (1,1) & (1,2) & (1,3) \\
			 2 & (2,1) & (2,2) & (2,3) \\
			 \hline			 
		\end{tabular}
		\qquad
		$\Rightarrow$
		\qquad
		\begin{tabular}{ccc|ccc}
			1 & 2 & 3 & 4 & 5 & 6\\
			\hline
			(1,1) & (1,2) & (1,3) & (2,1) & (2,2) & (2,3)
		\end{tabular}\,.
	\end{center}
	You can see an example for a \textbf{core-core matrix} with NEQ=2 and NNL=3 in Tab.~\ref{tab:example}.
	The formula for the common index $k$ is
	\begin{equation}
		k = N_L \,(i-1) + j\,,
		\label{fromNNLNCHtoINDEX}
	\end{equation}
	where $N_L=\mtm{NNL}$.
	
	
	In the case of an \textbf{asymptotic-core matrix} the core has NEQ$\times$NNL elements while the asymptotic has NEQ elements and therefore we have something similar to Tab.~\ref{tab:example2}.
	Finally in the case of an \textbf{asymptotic-asymptotic matrix} we have only NCH$\times$NCH elements as in Tab.~\ref{tab:example3}
	 \begin{table}[t]
	 	\centering
	 	\renewcommand{\arraystretch}{1.3}
	 	\setlength{\tabcolsep}{6pt}
	 	\begin{tabular}{|ccc|ccc|}
	 		\hline
	 		\textcolor{red}{(F1,1)} & \textcolor{red}{(F1,2)} & \textcolor{red}{(F1,3)} & \textcolor{blue}{(F1,4)} & \textcolor{blue}{(F1,5)} & \textcolor{blue}{(F1,6)} \\\hline
	 		\textcolor{green!60!black}{(F2,1)} & \textcolor{green!60!black}{(F2,2)} & \textcolor{green!60!black}{(F2,3)} & \textcolor{cyan!80!black}{(F2,4)} & \textcolor{cyan!80!black}{(F2,5)} & \textcolor{cyan!80!black}{(F2,6)} \\
	 		\hline
	 	\end{tabular}
	 	\caption{
	 		Example of a block structure of the $(NCH) \times (NNL \times NCH)$ matrix for $NCH=2$, $NNL=3$.
	 		Each entry is of the form $(\mathrm{index}_\mathrm{row}, \mathrm{index}_\mathrm{col})$.
	 		Text color highlights the four $3\times3$ channel ($\alpha$, $\alpha'$) blocks: (1,1) in red, (1,2) in blue, (2,1) in green, (2,2) in cyan.
	 	}
	 	\label{tab:example2}
	 \end{table}
	 \begin{table}[t]
	 	\centering
	 	\renewcommand{\arraystretch}{1.3}
	 	\setlength{\tabcolsep}{6pt}
	 	\begin{tabular}{|c|c|}
	 		\hline
	 		\textcolor{red}{(F1,F1)} & \textcolor{red}{(F1,F2)}  \\\hline
	 		\textcolor{green!60!black}{(F2,F1)} & \textcolor{green!60!black}{(F2,F2)}\\
	 		\hline
	 	\end{tabular}
	 	\caption{
	 		Example of a block structure of the $(NCH) \times (NCH)$ matrix for $NCH=2$.
	 		Each entry is of the form $(\mathrm{index}_\mathrm{row}, \mathrm{index}_\mathrm{col})$.
	 		Text color highlights the four $3\times3$ channel ($\alpha$, $\alpha'$) blocks: (1,1) in red, (1,2) in blue, (2,1) in green, (2,2) in cyan.
	 	}
	 	\label{tab:example3}
	 \end{table}
	 
	 \subsection*{Inverting the indexes}
	 If we know the the value of NNL we can retrieve the channel and the Laguerre $n$ value.
	 Reversing \eq{\eqref{fromNNLNCHtoINDEX}} we get
	 \begin{equation}
	 	\begin{cases}
	 		i &= \left\lfloor\dfrac{k-1}{N_L}\right\rfloor+1\,,\\
	 		j &= [(k-1)\mod N_L ]+1\,.
	 	\end{cases}
	 \end{equation}
	 It is impossible though to do the same knowing only the column number NEQ.
	 
	 \section{Separating the potential integration and reassembling it}
	 Here is a hopefully comprehensive description of the variational method used to fit a potential. Pre-evaluating a grid for all the radial functions of the potential and for all channels and then reassembling it changing the low-energy constants (LECs) of the potential.
	 \\
	 The potential is written as
	 \begin{equation}
	 	\hat{V}(r) = \sum_{o=\{\mtm{LO},\mtm{NLO},\mtm{N3LO}\}}\sum_{i=1}^{N_o},\
	 	C_i^{(o)}\,\hat{O}_i\,
	 	v_i^{(o)}(r)\,.
	 	\label{potential_expanded}
	 \end{equation}
	 Where $N_o$ is the number of operators per potential order: 1 for LO, 8 for NLO and 14 for N3LO. 
	 \begin{center}
	 	\begin{tabular}{c|l}
	 		\hline
	 		order & terms\\\hline
	 		LO & c \\
	 		NLO & c, $\tau$, $\sigma$, $\sigma\tau$, 
	 		$t$, $t\tau$, $b$, $T$\\
	 		N3LO & c, $\tau$, $\sigma$, $\sigma\tau$, 
	 		$t$, $t\tau$, $b$, $b\tau$, $bb$, $q$, $q\sigma$, $T$, $\sigma T$, $t T$, $b T$\\
	 		\hline
	 	\end{tabular}
	 \end{center}
	 Despite this number of operators, the number of radial functions is only 8, I call them $f_i(r)$ with $i=\{0,7\}$. Here
	 \begin{equation}
	 	f_0(r) \equiv C_R(r) = \frac{1}{\pi^{3/2}\,R^3}\,e^{-r^2/R^2}\,,
	 \end{equation}
	 where $R\equiv R_{ST}$ is the cutoff depending on the spin $S$ and isospin $T$. The other functions can be written as
	 \begin{equation}
	 	f_{i}(r) = g_i(r,R_{ST})\,f_0(r)\,,
	 \end{equation}
	 where $g_i(r,R)$ is a polynomial in $r$ and $g_0 = 1$.\\
	 \newcommand{\LO}{\mtm{LO}}
	 \newcommand{\NLO}{\mtm{NLO}}
	 \newcommand{\NNNLO}{\mtm{N3LO}}
	 \begin{table}[t]
	 	\centering
	 	\renewcommand{\arraystretch}{2.3} 
	 	\begin{tabular}{|c|c|l|l|l|}
	 		\hline
	 		\rowcolor{yellow}
	 		$f_i(r)$ & $g_i(r,R)$ & LO & NLO & N3LO \\\hline
	 		$f_0(r)$ & 1 & $v_c$ & $v_T$ & \\
	 		$f_1(r)$ & $\dfrac{6R^2-4r^2}{R^4}$ &  & $v_c$, $v_\tau$, $v_\sigma$, $v_{\sigma\tau}$ & $v_T$, $v_{\sigma T}$\\
	 		$f_2(r)$ & $-\dfrac{4r^2}{R^4}$ & & $v_t$, $v_{t\tau}$ &  $v_{tT}$\\
	 		$f_3(r)$ & $\dfrac{2}{R^2}$ &  & $v_b$ & $v_{bT}$\\
	 		$f_4(r)$ & $4\,\dfrac{4R^4-20R^2r^2+15r^4}{R^8}$ &  & & $v_c$, $v_\tau$, $v_\sigma$, $v_{\sigma\tau}$\\
	 		$f_5(r)$ & $8r^2\,\dfrac{2r^2-7R^2}{R^8}$ & & & $v_t$, $v_{t\tau}$\\
	 		$f_6(r)$ & $4\,\dfrac{5R^2-2r^2}{R^6}$ & & & $v_b$, $v_{b\tau}$\\
	 		$f_7(r)$ & $-\dfrac{4}{R^4}$ & & & $v_{bb}$, $v_q$, $v_{q\sigma}$,\\
	 		\hline
	 	\end{tabular}
	 	\caption{The radial functions $v_i^{(o)}(r)$ associated with $f_i(r)$.}
	 	\label{tab:functions}
	 \end{table}
	 A comprehensive connection between the $f_i$ and the $v_i^{(o)}$ as well as the value of each $g_i$ is given in Tab.~\ref{tab:functions}.
	 
	 
	 \subsection{The potential matrix elements}
	 In the program we need matrix elements of the type
	 \begin{equation}
	 	\matrixelement{\psi_\alpha}{\hat{V}(r)}{\psi_{\alpha'}}\,,
	 \end{equation}
	 where $\psi_\alpha$ can be a combination of core/asymptotic and regular/irregular. 
	 The core functions for a specific channel are expanded as
	 \begin{equation}
	 	\psi_\alpha^c(r) = \sum_{n=0}^{N_L}\,\mathcal{L}_n(\gamma r)\ket{\alpha}\equiv \sum_n\,\ket{n\alpha}\,
	 \end{equation}
	 Where the $\mathcal{L}_n = A\,L_n^{(2)}(\gamma r)$ are normalized Laguerre functions. 
	 The asymptotic functions are the modified Bessel functions 
	 \begin{equation}
	 	\psi_\alpha^{a, \mtm{R}} = \tilde{F}_L(kr)\,\ket{\alpha} \equiv \ket{F_\alpha}
	 	\mand
	 	\psi_\alpha^{a, \mtm{I}} = \tilde{G}_L(kr)\,\ket{\alpha}\equiv \ket{G_\alpha}\,,
	 \end{equation}
	 where R (I) stands for regular (irregular) and the modification is only for small values of $r$ and $k$, making them well behaved for these small values, but keeping them asymptotically indistinguishable (apart from a normalization constant) from $j_L(r)$ and $y_L(r)$.
	 
	 We have therefore 7 types of integrals to consider
	 \begin{align}
	 	\matrixelement{n'\alpha'}{\hat{V}(r)}{n\alpha} &= \int \de{r}r^2\,\mathcal{L}_{n'}(\gamma r)\,
	 	\mathcal{L}_{n}(\gamma r)\,
	 	\matrixelement{\alpha'}{\hat{V}(r)}{\alpha}\,,\\
	 	\matrixelement{F_{\alpha'}}{\hat{V}(r)}{n\alpha} &= \int \de{r}r^2\,
	 	F_{\alpha'}(kr)
	 	\,\mathcal{L}_n(\gamma r)\,
	 	\matrixelement{\alpha'}{\hat{V}(r)}{\alpha}\,,\\
	 	\matrixelement{G_{\alpha'}}{\hat{V}(r)}{n\alpha} &= \int \de{r}r^2\,
	 	G_{\alpha'}(kr)
	 	\,\mathcal{L}_n(\gamma r)\,
	 	\matrixelement{\alpha'}{\hat{V}(r)}{\alpha}\,,\\
	 	\matrixelement{F_{\alpha'}}{\hat{V}(r)}{F_\alpha} &= \int \de{r}r^2\,F_{\alpha'}(kr)\,
	 	F_{\alpha}(kr)\,
	 	\matrixelement{\alpha'}{\hat{V}(r)}{\alpha}\,,\\
	 	\matrixelement{F_{\alpha'}}{\hat{V}(r)}{G_\alpha} &= \int \de{r}r^2\,F_{\alpha'}(kr)\,
	 	G_{\alpha}(kr)\,
	 	\matrixelement{\alpha'}{\hat{V}(r)}{\alpha}\,,\\
	 	\matrixelement{G_{\alpha'}}{\hat{V}(r)}{F_\alpha} &= \int \de{r}r^2\,G_{\alpha'}(kr)\,
	 	F_{\alpha}(kr)\,
	 	\matrixelement{\alpha'}{\hat{V}(r)}{\alpha}\,,\\
	 	\matrixelement{G_{\alpha'}}{\hat{V}(r)}{G_\alpha} &= \int \de{r}r^2\,G_{\alpha'}(kr)\,
	 	G_{\alpha}(kr)\,
	 	\matrixelement{\alpha'}{\hat{V}(r)}{\alpha}\,.
	 \end{align}
	 The first is an integral of the kind \textbf{core-core}, the second and the third are of the kind \textbf{asymptotic-core} and the last four of the kind \textbf{asymptotic-asymptotic}.\\
	 Notice that, apart from the core-core ones, all the other integrals \textbf{depend on the scattering energy} through the momentum $k$.
	 
	 \subsection{Expansion of the potential in the integrals}
	 We saw that the integrals have the form
	 \begin{equation}
	 	\matrixelement{A\alpha}{\hat{V}(r)}{B\beta}
	 	=
	 	\int \de{r}r^2\,A(r)\,
	 	B(r)\,
	 	\matrixelement{\alpha}{\hat{V}(r)}{\beta}\,.
	 \end{equation}
	 Since we need to do this kind of integrals thousands if not hundreds of thousands of times we want to store the most essential information we can.\\
	 We start by expanding the potential as in \eq{\eqref{potential_expanded}}, obtaining
	 \begin{equation}
	 	\matrixelement{A\alpha}{\hat{V}(r)}{B\beta}
	 	=
	 	\sum_{o=\{\mtm{LO},\mtm{NLO},\mtm{N3LO}\}}\sum_{i=1}^{N_o}\
	 	C_i^{(o)}\,
	 	\matrixelement{\alpha}{\hat{O}_i}{\beta}
	 	\int \de{r}r^2\,A_\alpha(r)\,
	 	B_\beta(r)\,v_i^{(o)}(r)
	 	\,.
	 \end{equation}
	 We know also that $v_i$ is one of the $f_i$ functions. Therefore the most basic integrals that we can perform once and store are of the kind
	 \begin{equation}
	 	\matrixelement{A_\alpha}{f_i}{B_\beta} =
	 	\int_0^{r_\mtm{max}}\,
	 	\de{r}r^2\,
	 	A_\alpha(r)\,
	 	f_i(r)\,
	 	B_\beta(r)\,.
	 	\label{generic_integrals}
	 \end{equation}
	 Apart from the core-core ones, these integrals depend on the scattering energy and on the channels $\alpha$ and $\beta$ and in general on the scattering channel, which defines what $\alpha$ and $\beta$ can be.
	 
	 \subsubsection{Example with $L=1$ and $S=1$ in $np$ scattering}
	 In this case the possible channels for a $np$ scattering are three, $^3P_0$, $^3P_1$ and the coupled $^3P_2-{^3}F_2$ channels. For each channel we are interested in 20 energies, going up to 1 MeV. We want for each energy and for each channel to find the scattering phase shifts and mixing angles. \\
	 We want to see how these change by changing the potential LECs, therefore we need to repeat this process $N=100$ times. Since the kinetic energy do not depend on the potential, we can evaluate it for each channel and energy once. 
	 Same goes for the integrals in \eq{\eqref{generic_integrals}}.
	 To store this information we need therefore several matrices. We declare them as 
	 \begin{verbatim}
	 	INTEGER, PARAMETER :: NCH = 3
	 	INTEGER, PARAMETER :: NEQ = 2
	 	INTEGER, PARAMETER :: NNL = 32
	 	INTEGER, PARAMETER :: NALPHA = NNL*NEQ
	 	DOUBLE PRECISION :: K_MINUS_E_CC   (NCH, NE, NALPHA, NALPHA)
	 	DOUBLE PRECISION :: K_MINUS_E_AC_R (NCH, NE, NEQ, NALPHA)
	 	DOUBLE PRECISION :: K_MINUS_E_AC_I (NCH, NE, NEQ, NALPHA)
	 	DOUBLE PRECISION :: K_MINUS_E_AA_RR(NCH, NE, NEQ, NEQ)
	 	DOUBLE PRECISION :: K_MINUS_E_AA_RI(NCH, NE, NEQ, NEQ)
	 	DOUBLE PRECISION :: K_MINUS_E_AA_IR(NCH, NE, NEQ, NEQ)
	 	DOUBLE PRECISION :: K_MINUS_E_AA_II(NCH, NE, NEQ, NEQ)
	 \end{verbatim}
	 These integrals are
	 \begin{align}
	 	\texttt{K\_MINUS\_E\_CC(ich, ie, nalphap, nalpha)} &= \matrixelement{n'\alpha'}{\hat{K}-E}{n\alpha}\,,\\
	 	\texttt{K\_MINUS\_E\_AC\_X(ich, ie, alphap, nalpha)} &= \matrixelement{X_{\alpha'}}{\hat{K}-E}{n\alpha}\,,\\
	 	\texttt{K\_MINUS\_E\_AA\_XY(ich, ie, alphap, alpha)} &= \matrixelement{X_{\alpha'}}{\hat{K}-E}{Y_\alpha}\,,
	 \end{align}
	 where $X,\,Y$ could be $F$ or $G$.
	 Here we have to evaluate
	 \begin{align}
	 	N_{ch} \,N_{E}\,\left(
	 	N_\alpha^2 +2 N_{eq}\,N_\alpha + 4 N_{eq}^2
	 	\right)
	 	&=
	 	N_{ch} \,N_{E}\,N_{eq}^2\left(
	 	N_L^2 +2 N_L + 4
	 	\right)\\
	 	&=
	 	3\times 20\times 2^2(32^2+2\times 32 +4)\\
	 	&= 262\,080~\mtm{integrals}\,.
	 \end{align}
	 Each integral has a different number of points, depending 
	 if it's a \textit{c-c}, an \textit{a-c} or an \textit{a-a }integral. The fact that we store them saves a factor $N=100$, because we won't do them every time we change potential.
	 
	 For the potential energy we can evaluate the integrals of $f_i$:
	 \begin{verbatim}
	 	DOUBLE PRECISION :: FMAT_CC   (0:7, NCH, NALPHA, NALPHA)
	 	DOUBLE PRECISION :: FMAT_AC_R (0:7, NCH, NE, NEQ, NALPHA)
	 	DOUBLE PRECISION :: FMAT_AC_I (0:7, NCH, NE, NEQ, NALPHA)
	 	DOUBLE PRECISION :: FMAT_AA_RR(0:7, NCH, NE, NEQ, NEQ)
	 	DOUBLE PRECISION :: FMAT_AA_RI(0:7, NCH, NE, NEQ, NEQ)
	 	DOUBLE PRECISION :: FMAT_AA_IR(0:7, NCH, NE, NEQ, NEQ)
	 	DOUBLE PRECISION :: FMAT_AA_II(0:7, NCH, NE, NEQ, NEQ)
	 \end{verbatim}
	 These integrals are
	 \begin{align}
	 \texttt{FMAT\_CC(i, ich, nalphap, nalpha)} &= \matrixelement{n'\alpha'}{f_i}{n\alpha}\,,\\
	 \texttt{FMAT\_AC\_X(i, ich, ie, alphap, nalpha)} &= \matrixelement{X_{\alpha'}}{f_i}{n\alpha}\,,\\
	 \texttt{FMAT\_AA\_XY(i, ich, ie, alphap, alpha)} &= \matrixelement{X_{\alpha'}}{f_i}{Y_\alpha}\,.
	 \end{align}
	 Once we have these integrands we can evaluate back the potential by combining them with the expectation value of the operators $\matrixelement{\alpha'}{\hat{O}_i}{\alpha}$ and the set of LECs we have chosen for that calculation and save them in 
	 \begin{verbatim}
	 	DOUBLE PRECISION :: VM_CC   (NCH, NE, NALPHA, NALPHA)
	 	DOUBLE PRECISION :: VM_AC_R (NCH, NE, NEQ, NALPHA)
	 	DOUBLE PRECISION :: VM_AC_I (NCH, NE, NEQ, NALPHA)
	 	DOUBLE PRECISION :: VM_AA_RR(NCH, NE, NEQ, NEQ)
	 	DOUBLE PRECISION :: VM_AA_RI(NCH, NE, NEQ, NEQ)
	 	DOUBLE PRECISION :: VM_AA_IR(NCH, NE, NEQ, NEQ)
	 	DOUBLE PRECISION :: VM_AA_II(NCH, NE, NEQ, NEQ)
	 \end{verbatim}
	 
	 
	 
\end{document}
