\documentclass[10pt,a4paper]{article}
\usepackage[latin1]{inputenc}
\usepackage[T1]{fontenc}
\usepackage{amsmath, amsfonts, amssymb, mathtools, amsthm, bm}
\usepackage{graphicx}
\usepackage{physics}
\usepackage{booktabs}
\usepackage{array}
\usepackage{appendix}
\usepackage{geometry}
\usepackage{hyperref}
\usepackage{xfrac}
\usepackage{empheq}
\usepackage{bibref}
\usepackage[most]{tcolorbox}
\geometry{left=2cm, right=2cm, top=3cm, bottom=3cm}

\author{Alessandro Grassi}
\title{Nucleon--Nucleon Scattering: \textit{R}- and \textit{S}-Matrix Formalism}

% --- Custom Boxes for Math Highlighting ---
\newtcbox{\mymathUtil}[1][]{nobeforeafter, math upper, tcbox raise base, enhanced,
	colframe=blue!30!black, colback=white!10, boxrule=1pt, #1}
\newtcbox{\mymath}[1][]{nobeforeafter, math upper, tcbox raise base, enhanced,
	colframe=blue!30!black, colback=blue!10, boxrule=1pt, #1}
\newtcbox{\mymathUsed}[1][]{nobeforeafter, math upper, tcbox raise base, enhanced,
	colframe=blue!30!black, colback=yellow!30, boxrule=1pt, #1}

% --- Bra-Ket notation ---
\DeclarePairedDelimiter\Bra{\langle}{\rvert}
\DeclarePairedDelimiter\Ket{\lvert}{\rangle}
\DeclarePairedDelimiterX\Braket[2]{\langle}{\rangle}{#1 \delimsize\vert #2}
\DeclarePairedDelimiter\wBra{\bigg\langle}{\bigg\rvert}
\DeclarePairedDelimiter\wKet{\bigg\lvert}{\bigg\rangle}
\DeclarePairedDelimiterX\wBraket[2]{\bigg\langle}{\bigg\rangle}{#1 \delimsize\vert #2}
\newcommand{\wbra}[1]{\wBra{#1}}
\newcommand{\wket}[1]{\wKet{#1}}
\newcommand{\wbraket}[2]{\wBraket{#1}{#2}}

% --- Shortcuts ---
\newcommand{\wmatrixelement}[3]{\wbra{#1}\,#2\,\wket{#3}}
\newcommand{\matrixelementred}[3]{\langle #1 \Vert #2 \Vert #3\rangle}
\newcommand{\CG}[3]{\braket{#1\,#2\,}{#3}}
\newcommand{\eq}[1]{Eq.~#1}
\newcommand{\de}[1]{\mathrm{d}#1\,}
\newcommand{\ve}[1]{\mathbf{#1}}
\newcommand{\onehalf}{\sfrac{1}{2}\,}
\newcommand{\ylm}[3]{Y_{#1}^{#2}(#3)}
\newcommand{\versor}[1]{\ve{\hat{#1}}}
\newcommand{\mand}{\qquad\text{and}\qquad}
\newcommand{\pd}[2]{\frac{\partial^{#1}}{\partial{#2}^{#1}}}
\newcommand{\mtm}[1]{\mathrm{#1}}

% --- Angular momentum symbols ---
\newcommand{\threejsymbol}[6]{
	\begin{pmatrix}
		#1 & #2 & #3\\
		#4 & #5 & #6
	\end{pmatrix}
}
\newcommand{\sixjsymbol}[6]{
	\begin{Bmatrix}
		#1 & #2 & #3\\
		#4 & #5 & #6
	\end{Bmatrix}
}

\mathtoolsset{showonlyrefs,showmanualtags}

\begin{document}
	\maketitle
	
	\tableofcontents
	
	\section{Introduction}
	
	In quantum scattering theory, the evolution of an interacting two-body system is elegantly encoded in the \textit{S}-matrix, while the \textit{R}-matrix offers a numerically convenient alternative based on boundary matching. This document reviews the definitions, physical interpretations, and mathematical derivations of both quantities, particularly within the context of nucleon--nucleon scattering.
	
	\section{The \textit{S}-Matrix and \textit{R}-Matrix: Definitions and Interpretations}
	
	\subsection{The Scattering Matrix (\textit{S}-Matrix)}
	
	\paragraph{Definition:}
	The \textit{S}-matrix connects asymptotic incoming and outgoing states:
	\[
	\Ket{\text{out}} = S \Ket{\text{in}}
	\]
	
	\paragraph{Physical Role:}
	\begin{itemize}
		\item Encodes all observable aspects of scattering: phase shifts, cross sections, and mixing.
		\item Ensures conservation of probability via unitarity: \( S^\dagger S = I \).
		\item For uncoupled channels: \( S_\ell = e^{2i\delta_\ell} \), where \( \delta_\ell \) is the phase shift.
	\end{itemize}
	
	\subsection{The Reactance Matrix (\textit{R}-Matrix)}
	
	\paragraph{Definition:}
	Defined via the logarithmic derivative of the wavefunction at the boundary of the interaction region:
	\[
	R_\ell(E) = \left. \frac{a\, u'_\ell(a)}{u_\ell(a)} \right|_{\text{internal}}
	\]
	
	\paragraph{Physical Role:}
	\begin{itemize}
		\item Arises from dividing configuration space into internal (\( r < a \)) and external (\( r > a \)) regions.
		\item Useful for resonance physics and numerical stability.
		\item Related to the \textit{S}-matrix via:
		\[
		S = \frac{1 + iR}{1 - iR}
		\]
	\end{itemize}
	
	\section{From Schr\"odinger Equation to Scattering Matrices}
	
	\subsection{Radial Schr\"odinger Equation}
	
	Consider two nucleons interacting via a central potential \( V(r) \). The time-independent Schr�dinger equation reads:
	\[
	\left[ -\frac{\hbar^2}{2\mu} \nabla^2 + V(r) \right] \psi(\ve{r}) = E \psi(\ve{r})
	\]
	
	\paragraph{Partial Wave Expansion:}
	Using spherical symmetry:
	\[
	\psi(\ve{r}) = \sum_{\ell m} \frac{u_\ell(r)}{r} Y_{\ell m}(\hat{r})
	\]
	The radial equation becomes:
	\[
	\left[-\frac{\hbar^2}{2\mu} \dv[2]{r} + \frac{\hbar^2 \ell(\ell+1)}{2\mu r^2} + V(r) \right] u_\ell(r) = E u_\ell(r)
	\]
	
	\paragraph{Asymptotic Behavior:}
	For \( r \to \infty \) (free motion), define the wave number \( k = \sqrt{2\mu E}/\hbar \). Then:
	\[
	u_\ell(r) \xrightarrow{r \to \infty} \sin\left(kr - \frac{\ell\pi}{2} + \delta_\ell\right)
	\]
	
	\subsection{Definition of the \textit{S}-Matrix}
	
	Rewriting the asymptotic form as a combination of incoming and outgoing spherical waves:
	\[
	u_\ell(r) \sim \frac{1}{2i} \left[ e^{-i(kr - \ell\pi/2)} - S_\ell \, e^{i(kr - \ell\pi/2)} \right]
	\]
	This identifies \( S_\ell = e^{2i\delta_\ell} \).
	
	\subsection{Definition of the \textit{R}-Matrix}
	
	In the R-matrix framework:
	\begin{itemize}
		\item \textbf{Internal region}: \( r < a \), where interactions occur.
		\item \textbf{External region}: \( r > a \), free-particle motion.
	\end{itemize}
	
	In the external region, the general solution is:
	\[
	u_\ell(r) = A_\ell \left[ F_\ell(kr) \cos\delta_\ell + G_\ell(kr) \sin\delta_\ell \right]
	\]
	Matching this with the internal solution at \( r = a \), one derives \cite{advances}:
	\[
	S_\ell = \frac{1 + iR_\ell}{1 - iR_\ell}
	\]
	
	\section{Matrix Structure: Uncoupled and Coupled Channels}
	
	\subsection{Summary Table: Key Characteristics}
	
	\begin{table}[h!]
		\centering
		\renewcommand{\arraystretch}{1.5}
		\begin{tabular}{| >{\centering\arraybackslash}m{2.5cm} | >{\centering\arraybackslash}m{2cm} | >{\centering\arraybackslash}m{5cm} | >{\centering\arraybackslash}m{5cm} |}
			\hline
			\textbf{Quantity} & \textbf{Uncoupled} & \textbf{Coupled (Stapp)} & \textbf{Coupled (BB)} \\
			\hline
			$S$ & 
			$e^{2i\delta}$ &  $\mathrm{diag}(e^{2i\delta_i})\,O(\epsilon) \,\mathrm{diag}(e^{2i\delta_i}) $ & 
			$O^T(\epsilon) \, \mathrm{diag}(e^{2i\delta_i}) \, O(\epsilon)$ \\
			\hline
			$R$ & 
			$\tan\delta$ & 
			$\mathrm{diag}(\tan\delta_i)\,
			O(\epsilon) \, \mathrm{diag}(\tan\delta_i)$ & 
			$O^T(\epsilon) \, \mathrm{diag}(\tan\delta_i) \, O(\epsilon)$ \\
			\hline
			$O$ & 
			-- & 
			\( O(\epsilon) = \begin{pmatrix} \cos2\epsilon & i\sin2\epsilon \\ i\sin2\epsilon & \cos2\epsilon \end{pmatrix} \) & 
			\( O(\epsilon) = \begin{pmatrix} \cos \epsilon & \sin \epsilon \\ -\sin \epsilon & \cos \epsilon \end{pmatrix} \) \\
			\hline
		\end{tabular}
		\caption{Summary of \textit{S}- and \textit{R}-matrix structures in uncoupled and coupled cases, using Stapp \cite{Stapp} and Blatt--Biedenharn \cite{BB} conventions.}
		\label{tab:s_r_matrices}
	\end{table}
	
	\subsection{Remarks}
	
	\begin{itemize}
		\item In coupled channels, the phase shifts \( \delta_i \) and mixing angles \( \epsilon \) fully characterize the scattering process.
		\item The two conventions differ in the placement of rotation matrices but yield the same observables.
	\end{itemize}
	
	
	\section{Variational code}
	In the variational code the $R$-matrix is evaluated using Ko\"on principle to second order. Then through the following steps one recovers the phase-shifts and mixing angles for both the Stapp and BB conventions.
	
	\subsection{BB phase shifts and mixing angle}
	Using Tab.~\ref{tab:s_r_matrices} the $R$ matrix can be written as
	\begin{equation}
		R= \left(
		\begin{array}{cc}
			\tan \delta_1 \cos ^2 \epsilon +\tan \delta_2 \sin ^2 \epsilon  & 
			\left(\tan\delta_1 -\tan\delta_2)\right) \sin\epsilon\, \cos  \epsilon \\
			\left(\tan\delta_1 -\tan\delta_2)\right) \sin\epsilon\, \cos  \epsilon & 
			\tan \delta_1 \sin ^2
			\epsilon +\tan \delta_2 \cos ^2\epsilon \\
		\end{array}
		\right)\,.
	\end{equation}
	The combination $R_{11}-R_{22}$ is
	\begin{equation}
		R_{11}-R_{22} = 
		\cos(2\epsilon) 
		\left(\tan \delta_1 - \tan \delta_2\right)\,.
	\end{equation}
	Therefore
	\begin{equation}
		\tan(4\epsilon) = \frac{2\,R_{12}}{R_{11}-R_{22}}
		\qquad
		\rightarrow
		\qquad
		\epsilon =
		\frac{1}{2} \atan\left(\frac{2\,R_{12}}{R_{11}-R_{22}}\right)\,.
	\end{equation}
	Once $\epsilon$ is known, one can evaluate 
	\begin{equation}
		\tan\delta_1
		=
		\cos^2 \epsilon \,\,R_{11}
		+\sin^2 \epsilon \,\,R_{22}
		+2 \cos\epsilon \,\sin\epsilon\, R_{12}
	\end{equation}
	and
	\begin{equation}
		\tan\delta_2
		=
		\sin^2 \epsilon\,\,R_{11}
		+\cos^2 \epsilon \,\,R_{22}
		-2 \cos\epsilon\, \sin\epsilon\, R_{12}\,.
	\end{equation}
	
	\subsection{Stapp phase shifts and mixing angle}
	One can then use $\delta_1$, $\delta_2$ and $\epsilon$ to evaluate  the $S$-matrix, which is independent from the parametrization,
	\begin{equation}
		S = S_\mtm{BB} = 
		\left(
		\begin{array}{cc}
			e^{2 i \delta _1} \cos ^2\epsilon + e^{2 i \delta _2} \sin ^2\epsilon & 
			\left(e^{2 i \delta _1} -e^{2 i \delta _2}\right) \sin \epsilon  \cos\epsilon \\
			\left(e^{2 i \delta _1} -e^{2 i \delta _2}\right) \sin \epsilon  \cos\epsilon &
			e^{2 i \delta _1} \sin ^2\epsilon+e^{2 i \delta _2} \cos
			^2\epsilon \\
		\end{array}
		\right)\,.
	\end{equation}
	It is possible now to extract the phase shifts and mixing angle in the Stapp parametrization.
	In this parametrization 
	\begin{equation}
		S = S_\mtm{Stapp} = 
		\left(
		\begin{array}{cc}
			e^{2 i \delta _1} \,\cos (2 \epsilon ) & i \,e^{i (\delta _1+\delta _2)}\, \sin (2 \epsilon ) \\
			i\, e^{i (\delta _1+\delta _2)}\, \sin (2 \epsilon ) & e^{2 i \delta _2} \cos (2 \epsilon ) \\
		\end{array}
		\right)\,.
	\end{equation}
	The determinant in this case is 
	\begin{equation}
		\det S_\mtm{Stapp} = e^{2 i \left(\delta _1+\delta _2\right)}
	\end{equation}
	and therefore
	\begin{equation}
		\sin(2\epsilon) = \sqrt{-\frac{S_{12}^2}{\det S}}
	\end{equation}
	and 
	\begin{equation}
		\cos(2\epsilon) = \sqrt{1-\sin^2(2\epsilon)}\,.
	\end{equation}
	It is possible to evaluate
	\begin{equation}
		e^{i\delta_k} = \sqrt{\frac{S_{kk}}{\cos(2\epsilon)}}
		=
		\sqrt{e^{2i\delta_k}}\,.
	\end{equation}
	Therefore
	\begin{equation}
		\delta_1 =
		\,\acos\left[\Re\left(\sqrt{\frac{S_{11}}{\cos(2\epsilon)}}\right)\right]
		\times
		\begin{cases}
			1 & \mathrm{if~}\Im\left(\sqrt{\frac{S_{11}}{\cos(2\epsilon)}}\right) \ge 0 \\
			-1 & \mathrm{if~}\Im\left(\sqrt{\frac{S_{11}}{\cos(2\epsilon)}}\right) < 0
		\end{cases}
	\end{equation}
	and 
	\begin{equation}
		\delta_2 =
		\acos\left[\Re\left(\frac{S_{22}}{\cos(2\epsilon)}\right)\right]
		\times
		\begin{cases}
			1 & \mathrm{if~}\Im\left(\frac{S_{22}}{\cos(2\epsilon)}\right) \ge 0 \\
			-1 & \mathrm{if~}\Im\left(\frac{S_{22}}{\cos(2\epsilon)}\right) < 0
		\end{cases}\,.
	\end{equation}
	
	
	
	\section{Conclusion}
	
	The \textit{S}-matrix and \textit{R}-matrix are central tools in analyzing nucleon--nucleon scattering. While the \textit{S}-matrix encapsulates the observable content of the interaction, the \textit{R}-matrix provides a convenient and often more numerically robust intermediate object, especially in resonance or coupled-channel analyses. Their connection through a M\"obius transformation reflects deep structural links in scattering theory.
	
	
	\begin{thebibliography}{9}
		\bibitem{advances}
		L.M. Delves, Advances in Nuclear Physics, vol.~5 (1972), Eds. M. Baranger, E. V\"ogt (Plenum Press, London, New York).
		
		\bibitem{Stapp}
		H.P. Stapp, T. Ypsilantis and N. Metropolis, Phys. Rev. \textbf{105}, 302 (1957).
		
		\bibitem{BB}
		J. M. Blatt and L. C. Biedenharn, Phys. Rev. \textbf{86}, 399 (1952).
		
		
	\end{thebibliography}
	
\end{document}
