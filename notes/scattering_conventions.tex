\documentclass[10pt,a4paper]{article}
\usepackage[latin1]{inputenc}
\usepackage[T1]{fontenc}
\usepackage{amsmath, amsfonts, amssymb, mathtools, amsthm, bm}
\usepackage{graphicx}
\usepackage{physics}
\usepackage{booktabs}
\usepackage{array}
\usepackage{appendix}
\usepackage{geometry}
\usepackage{hyperref}
\usepackage{xfrac}
\usepackage{empheq}
\usepackage{bibref}
\usepackage[most]{tcolorbox}
\usepackage{cancel}
\geometry{left=2cm, right=2cm, top=3cm, bottom=3cm}

\author{Alessandro Grassi}
\title{Nucleon--Nucleon Scattering: \textit{R}- and \textit{S}-Matrix Formalism}

% --- Custom Boxes for Math Highlighting ---
\newtcbox{\mymathUtil}[1][]{nobeforeafter, math upper, tcbox raise base, enhanced,
	colframe=blue!30!black, colback=white!10, boxrule=1pt, #1}
\newtcbox{\mymath}[1][]{nobeforeafter, math upper, tcbox raise base, enhanced,
	colframe=blue!30!black, colback=blue!10, boxrule=1pt, #1}
\newtcbox{\mymathUsed}[1][]{nobeforeafter, math upper, tcbox raise base, enhanced,
	colframe=blue!30!black, colback=yellow!30, boxrule=1pt, #1}

% --- Bra-Ket notation ---
\DeclarePairedDelimiter\Bra{\langle}{\rvert}
\DeclarePairedDelimiter\Ket{\lvert}{\rangle}
\DeclarePairedDelimiterX\Braket[2]{\langle}{\rangle}{#1 \delimsize\vert #2}
\DeclarePairedDelimiter\wBra{\bigg\langle}{\bigg\rvert}
\DeclarePairedDelimiter\wKet{\bigg\lvert}{\bigg\rangle}
\DeclarePairedDelimiterX\wBraket[2]{\bigg\langle}{\bigg\rangle}{#1 \delimsize\vert #2}
\newcommand{\wbra}[1]{\wBra{#1}}
\newcommand{\wket}[1]{\wKet{#1}}
\newcommand{\wbraket}[2]{\wBraket{#1}{#2}}

% --- Shortcuts ---
\newcommand{\wmatrixelement}[3]{\wbra{#1}\,#2\,\wket{#3}}
\newcommand{\matrixelementred}[3]{\langle #1 \Vert #2 \Vert #3\rangle}
\newcommand{\CG}[3]{\braket{#1\,#2\,}{#3}}
\newcommand{\eq}[1]{Eq.~#1}
\newcommand{\de}[1]{\mathrm{d}#1\,}
\newcommand{\ve}[1]{\mathbf{#1}}
\newcommand{\onehalf}{\sfrac{1}{2}\,}
\newcommand{\ylm}[3]{Y_{#1}^{#2}(#3)}
\newcommand{\versor}[1]{\ve{\hat{#1}}}
\newcommand{\mand}{\qquad\text{and}\qquad}
\newcommand{\pd}[2]{\frac{\partial^{#1}}{\partial{#2}^{#1}}}
\newcommand{\mtm}[1]{\mathrm{#1}}

% --- Angular momentum symbols ---
\newcommand{\threejsymbol}[6]{
	\begin{pmatrix}
		#1 & #2 & #3\\
		#4 & #5 & #6
	\end{pmatrix}
}
\newcommand{\sixjsymbol}[6]{
	\begin{Bmatrix}
		#1 & #2 & #3\\
		#4 & #5 & #6
	\end{Bmatrix}
}

\mathtoolsset{showonlyrefs,showmanualtags}

\begin{document}
	\maketitle
	
	\tableofcontents
	
	\section{Introduction}
	
	In quantum scattering theory, the evolution of an interacting two-body system is elegantly encoded in the \textit{S}-matrix, while the \textit{R}-matrix offers a numerically convenient alternative based on boundary matching. This document reviews the definitions, physical interpretations, and mathematical derivations of both quantities, particularly within the context of nucleon--nucleon scattering.
	
	\section{The \textit{S}-Matrix and \textit{R}-Matrix: Definitions and Interpretations}
	
	\subsection{The Scattering Matrix (\textit{S}-Matrix)}
	
	\paragraph{Definition:}
	The \textit{S}-matrix connects asymptotic incoming and outgoing states:
	\[
	\Ket{\text{out}} = S \Ket{\text{in}}
	\]
	
	\paragraph{Physical Role:}
	\begin{itemize}
		\item Encodes all observable aspects of scattering: phase shifts, cross sections, and mixing.
		\item Ensures conservation of probability via unitarity: \( S^\dagger S = I \).
		\item For uncoupled channels: \( S_\ell = e^{2i\delta_\ell} \), where \( \delta_\ell \) is the phase shift.
	\end{itemize}
	
	\subsection{The Reactance Matrix (\textit{R}-Matrix)}
	
	\paragraph{Definition:}
	Defined via the logarithmic derivative of the wavefunction at the boundary of the interaction region:
	\[
	R_\ell(E) = \left. \frac{a\, u'_\ell(a)}{u_\ell(a)} \right|_{\text{internal}}
	\]
	
	\paragraph{Physical Role:}
	\begin{itemize}
		\item Arises from dividing configuration space into internal (\( r < a \)) and external (\( r > a \)) regions.
		\item Useful for resonance physics and numerical stability.
		\item Related to the \textit{S}-matrix via:
		\[
		S = \frac{1 + iR}{1 - iR}
		\]
	\end{itemize}
	
	\section{From Schr\"odinger Equation to Scattering Matrices}
	
	\subsection{Radial Schr\"odinger Equation}
	
	Consider two nucleons interacting via a central potential \( V(r) \). The time-independent Schr\"odinger equation reads:
	\[
	\left[ -\frac{\hbar^2}{2\mu} \nabla^2 + V(r) \right] \psi(\ve{r}) = E \psi(\ve{r})
	\]
	
	\paragraph{Partial Wave Expansion:}
	Using spherical symmetry:
	\[
	\psi(\ve{r}) = \sum_{\ell m} \frac{u_\ell(r)}{r} Y_{\ell m}(\hat{r})
	\]
	The radial equation becomes:
	\[
	\left[-\frac{\hbar^2}{2\mu} \dv[2]{r} + \frac{\hbar^2 \ell(\ell+1)}{2\mu r^2} + V(r) \right] u_\ell(r) = E u_\ell(r)
	\]
	
	\paragraph{Asymptotic Behavior:}
	For \( r \to \infty \) (free motion), define the wave number \( k = \sqrt{2\mu E}/\hbar \). Then:
	\[
	u_\ell(r) \xrightarrow{r \to \infty} \sin\left(kr - \frac{\ell\pi}{2} + \delta_\ell\right)
	\]
	
	\subsection{Definition of the \textit{S}-Matrix}
	
	Rewriting the asymptotic form as a combination of incoming and outgoing spherical waves:
	\[
	u_\ell(r) \sim \frac{1}{2i} \left[ e^{-i(kr - \ell\pi/2)} - S_\ell \, e^{i(kr - \ell\pi/2)} \right]
	\]
	This identifies \( S_\ell = e^{2i\delta_\ell} \).
	
	\subsection{Definition of the \textit{R}-Matrix}
	
	In the R-matrix framework:
	\begin{itemize}
		\item \textbf{Internal region}: \( r < a \), where interactions occur.
		\item \textbf{External region}: \( r > a \), free-particle motion.
	\end{itemize}
	
	In the external region, the general solution is:
	\begin{equation}
		\psi_\ell(r) = C_\ell\, \left[ F_\ell(kr) \cos\delta_\ell + G_\ell(kr) \sin\delta_\ell \right]\,,
		\label{eq:asymptotic_u}
	\end{equation}
	where $C_\ell$ is a normalization constant.
	Matching this with the internal solution at \( r = a \), one derives \cite{advances}:
	\begin{equation}
		S_\ell = \frac{1 + iR_\ell}{1 - iR_\ell}
	\end{equation}
	
	\section{Matrix Structure: Uncoupled and Coupled Channels}
	
	\subsection{Summary Table: Key Characteristics}
	
	\begin{table}[h!]
		\centering
		\renewcommand{\arraystretch}{1.5}
		\begin{tabular}{| >{\centering\arraybackslash}m{2.cm} | >{\centering\arraybackslash}m{2cm} | >{\centering\arraybackslash}m{5cm} | >{\centering\arraybackslash}m{5cm} |}
			\hline
			\textbf{Quantity} & \textbf{Uncoupled} & \textbf{Coupled (Stapp)} & \textbf{Coupled (BB)} \\
			\hline
			$S$ & 
			$e^{2i\delta}$ &  $\mathrm{diag}(e^{2i\delta_i})\,O(\epsilon) \,\mathrm{diag}(e^{2i\delta_i}) $ & 
			$O^T(\epsilon) \, \mathrm{diag}(e^{2i\delta_i}) \, O(\epsilon)$ \\
			\hline
			$R$ & 
			$\tan\delta$ & 
			$\mathrm{diag}(\tan\delta_i)\,
			O(\epsilon) \, \mathrm{diag}(\tan\delta_i)$ & 
			$O^T(\epsilon) \, \mathrm{diag}(\tan\delta_i) \, O(\epsilon)$ \\
			\hline
			$O$ & 
			-- & 
			\( O(\epsilon) = \begin{pmatrix} \cos2\epsilon & i\sin2\epsilon \\ i\sin2\epsilon & \cos2\epsilon \end{pmatrix} \) & 
			\( O(\epsilon) = \begin{pmatrix} \cos \epsilon & \sin \epsilon \\ -\sin \epsilon & \cos \epsilon \end{pmatrix} \) \\
			\hline
		\end{tabular}
		\caption{Summary of \textit{S}- and \textit{R}-matrix structures in uncoupled and coupled cases, using Stapp \cite{Stapp} and Blatt--Biedenharn \cite{BB} conventions.}
		\label{tab:s_r_matrices}
	\end{table}
	
	\subsection{Remarks}
	
	\begin{itemize}
		\item In coupled channels, the phase shifts \( \delta_i \) and mixing angles \( \epsilon \) fully characterize the scattering process.
		\item The two conventions differ in the placement of rotation matrices but yield the same observables.
	\end{itemize}
	
	
	\section{Variational code}
	In the variational code the $R$-matrix is evaluated using Ko\"on principle to second order. Then through the following steps one recovers the phase-shifts and mixing angles for both the Stapp and the Blatt--Biedenharn (BB) conventions.
	
	\subsection{BB phase shifts and mixing angle}
	Using Tab.~\ref{tab:s_r_matrices} the $R$ matrix can be written as
	\begin{equation}
		R= \left(
		\begin{array}{cc}
			\tan \delta_1 \cos ^2 \epsilon +\tan \delta_2 \sin ^2 \epsilon  & 
			\left(\tan\delta_1 -\tan\delta_2)\right) \sin\epsilon\, \cos  \epsilon \\
			\left(\tan\delta_1 -\tan\delta_2)\right) \sin\epsilon\, \cos  \epsilon & 
			\tan \delta_1 \sin ^2
			\epsilon +\tan \delta_2 \cos ^2\epsilon \\
		\end{array}
		\right)\,.
		\label{eq:RBB}
	\end{equation}
	The combination $R_{11}-R_{22}$ is
	\begin{equation}
		R_{11}-R_{22} = 
		\cos(2\epsilon) 
		\left(\tan \delta_1 - \tan \delta_2\right)\,.
	\end{equation}
	Therefore
	\begin{equation}
		\tan(4\epsilon) = \frac{2\,R_{12}}{R_{11}-R_{22}}
		\qquad
		\rightarrow
		\qquad
		\epsilon =
		\frac{1}{2} \atan\left(\frac{2\,R_{12}}{R_{11}-R_{22}}\right)\,.
	\end{equation}
	Once $\epsilon$ is known, one can evaluate 
	\begin{equation}
		\tan\delta_1
		=
		\cos^2 \epsilon \,\,R_{11}
		+\sin^2 \epsilon \,\,R_{22}
		+2 \cos\epsilon \,\sin\epsilon\, R_{12}
	\end{equation}
	and
	\begin{equation}
		\tan\delta_2
		=
		\sin^2 \epsilon\,\,R_{11}
		+\cos^2 \epsilon \,\,R_{22}
		-2 \cos\epsilon\, \sin\epsilon\, R_{12}\,.
	\end{equation}
	
	\subsection{Stapp phase shifts and mixing angle}
	One can then use $\delta_1$, $\delta_2$ and $\epsilon$ to evaluate  the $S$-matrix, which is independent from the parametrization,
	\begin{equation}
		S = S_\mtm{BB} = 
		\left(
		\begin{array}{cc}
			e^{2 i \delta _1} \cos ^2\epsilon + e^{2 i \delta _2} \sin ^2\epsilon & 
			\left(e^{2 i \delta _1} -e^{2 i \delta _2}\right) \sin \epsilon  \cos\epsilon \\
			\left(e^{2 i \delta _1} -e^{2 i \delta _2}\right) \sin \epsilon  \cos\epsilon &
			e^{2 i \delta _1} \sin ^2\epsilon+e^{2 i \delta _2} \cos
			^2\epsilon \\
		\end{array}
		\right)\,.
		\label{SBB}
	\end{equation}
	It is possible now to extract the phase shifts and mixing angle in the Stapp parametrization.
	In this parametrization 
	\begin{equation}
		S = S_\mtm{Stapp} = 
		\left(
		\begin{array}{cc}
			e^{2 i \delta _1} \,\cos (2 \epsilon ) & i \,e^{i (\delta _1+\delta _2)}\, \sin (2 \epsilon ) \\
			i\, e^{i (\delta _1+\delta _2)}\, \sin (2 \epsilon ) & e^{2 i \delta _2} \cos (2 \epsilon ) \\
		\end{array}
		\right)\,.
	\end{equation}
	The determinant in this case is 
	\begin{equation}
		\det S_\mtm{Stapp} = e^{2 i \left(\delta _1+\delta _2\right)}
	\end{equation}
	and therefore
	\begin{equation}
		\sin(2\epsilon) = \sqrt{-\frac{S_{12}^2}{\det S}}
		\label{eq:epsStapp}
	\end{equation}
	and 
	\begin{equation}
		\cos(2\epsilon) = \sqrt{1-\sin^2(2\epsilon)}\,.
	\end{equation}
	It is possible to evaluate
	\begin{equation}
		e^{i\delta_k} = \sqrt{\frac{S_{kk}}{\cos(2\epsilon)}}
		=
		\sqrt{e^{2i\delta_k}}\,.
	\end{equation}
	Therefore
	\begin{equation}
		\delta_1 =
		\,\acos\left[\Re\left(\sqrt{\frac{S_{11}}{\cos(2\epsilon)}}\right)\right]
		\times
		\begin{cases}
			1 & \mathrm{if~}\Im\left(\sqrt{\frac{S_{11}}{\cos(2\epsilon)}}\right) \ge 0 \\
			-1 & \mathrm{if~}\Im\left(\sqrt{\frac{S_{11}}{\cos(2\epsilon)}}\right) < 0
		\end{cases}
	\end{equation}
	and 
	\begin{equation}
		\delta_2 =
		\acos\left[\Re\left(\frac{S_{22}}{\cos(2\epsilon)}\right)\right]
		\times
		\begin{cases}
			1 & \mathrm{if~}\Im\left(\frac{S_{22}}{\cos(2\epsilon)}\right) \ge 0 \\
			-1 & \mathrm{if~}\Im\left(\frac{S_{22}}{\cos(2\epsilon)}\right) < 0
		\end{cases}\,.
	\end{equation}
	
	
	
		\section{Asymptotic expansion at low energies}
		As seen in \eq{\eqref{eq:asymptotic_u}} it is possibly to write for non-coupled channels
		\begin{equation}
			\psi_\ell(r) = 
			A_\ell\,
			\left(
				F_\ell
				+ \tan \delta_\ell \,
				G_\ell
			\right)\,,
		\end{equation}
		where $A_\ell \equiv C_\ell\, \cos \delta_\ell$.
		In this case $R_{\ell \ell } = \tan \delta_\ell$ and therefore
		\begin{equation}
			\psi_\ell(r) = 
			A_\ell\,
			\left(
			F_\ell
			+ R_{\ell\ell} \,
			G_\ell
			\right)\,.
		\end{equation}
		Here $R_\ell$ ($G_\ell$) are the regular (irregular) solution to the Schr\"odinger equation and in the case if the potential is short-range 
		\begin{equation}
			\begin{cases}
				F_\ell(kr) &\rightarrow\quad j_\ell(kr)\,,\\
				G_\ell(kr) &\rightarrow\quad y_\ell(kr)\,,\\
			\end{cases}
		\end{equation}
		where $j_\ell$ ($y_\ell$) is the regular (irregular) spherical Bessel function.\\
		In case of a long range potential of the type $\propto 1/r$ 
		\begin{equation}
			\begin{cases}
				F_\ell & \rightarrow \quad F_\ell(\eta, kr)\,,\\
				G_\ell & \rightarrow \quad G_\ell(\eta, kr)\,,
			\end{cases}
		\end{equation}
		where $F_\ell(\eta,kr)$ ($G_\ell(\eta,kr)$) is the regular (irregular) Coulomb function.
		
		\subsection{The scattering length}
		For small energies there are two observables of importance, which depend linearly on the energy, the scattering length $a_\ell$ and the effective range $r_{e,\ell}$, they are connected to the momentum $k$ and the phase shift $\delta_\ell$ by 
		\begin{equation}
			k^{2\ell+1}\,\cot\delta_\ell = -\frac{1}{a_\ell} + \frac{1}{2}\,r_{e,\ell}\,k^2\,. 
		\end{equation}
		Specifically, for $E\rightarrow 0$ 
		\begin{equation}
			k^{2\ell+1}\,\cot\delta_\ell\rightarrow -\frac{1}{a_\ell}\,.
		\end{equation}
		Since for uncoupled channels $R_{\ell\ell}=\tan\delta_\ell$ it is possible to write
		\begin{equation}
			R_{\ell\ell} \rightarrow 
			-a_\ell\,k^{2\ell+1}\,.
		\end{equation}
		
		\subsubsection{Eliminating the energy dependence from the spherical Bessel functions}
		Focusing on the spherical
		Bessel function, the asymptotic limit for their argument going to zero is \cite{HS}
		\begin{equation}
			\begin{cases}
				F_\ell(kr) = j_\ell(x) &\rightarrow
				\quad \dfrac{x^\ell}{(2\ell+1)!!}\,,\\[2.ex]
				G_\ell(kr)=y_\ell(x) &\rightarrow
				\quad- \dfrac{(2\ell-1)!!}{x^{\ell+1}}\,.
			\end{cases}
		\end{equation}
		Using for $E\rightarrow 0$ the functions [code]
		\begin{equation}
			\begin{cases}
				\tilde{F}_\ell( r) = r^\ell\,, \\
				\tilde{G}_\ell( r) = -\dfrac{1}{(2\ell + 1)\, r^{\ell+1}} \left(1 - e^{-\epsilon r}\right)^{2\ell + 1}\,,
			\end{cases}
		\end{equation}
		for $r\rightarrow \infty$ it is possible to write
		\begin{equation}
			\tilde{A}_\ell
			\left(\tilde{F}_\ell+R\,\tilde{G}_\ell
			\right)
			=
			\tilde{A}_\ell
			\left(r^\ell
			-
			\dfrac{\tilde{R}_{\ell\ell}}{2\ell+1}\,\frac{1}{r^{\ell+1}}\right)
		\end{equation}
		and confronting it with the actual behavior
		\begin{equation}
			A_\ell\,
			\left(
			F_\ell(kr)
			+
			R_{\ell\ell}\,
			G_\ell(kr)
			\right)
			\simeq
			A_\ell
			\left(
			\frac{k^\ell r^\ell}{(2\ell+1)!!}
			-
			R_{\ell\ell}\,
			\frac{(2\ell-1)!!}{k^{\ell+1}r^{\ell+1}}
			\right)
		\end{equation}
		it is possible to map
		\begin{equation}
			\dfrac{\tilde{R}_{\ell\ell}}{2\ell+1}
			=
			\frac{(2\ell+1)!!\,(2\ell-1)!!}{k^{2\ell+1}}\,R_{\ell\ell}
			\label{eq:connectionRs}
		\end{equation}
		and therefore
		\begin{equation}
			\tilde{R}_{\ell\ell} = -
			\left[(2\ell+1)!!\right]^2\,
			a_\ell\,.
		\end{equation}
		
		\paragraph{Coupled case}
		In the case of a coupled channel
		\begin{equation}
			\begin{cases}
				A_{\ell_1} \left(F_{\ell_1}(kr) + R_{\ell_1\ell_1}\,G_{\ell_1}(kr) + 
				R_{\ell_1(\ell_2)}\,G_{\ell_2}(kr)\right)\,,\\
				A_{\ell_2} \left(F_{\ell_2}(kr) + R_{(\ell_2)\ell_1}\,G_{\ell_1}(kr) + 
				R_{(\ell_2)(\ell_2)}\,G_{\ell_2}(kr)\right)
			\end{cases}
		\end{equation}
		the choice for small energies brings to
		\begin{equation}
			\begin{cases}
				\tilde{A}_{\ell_1}
				\left(
				r^{\ell_1} - \dfrac{\tilde{R}_{\ell_1\ell_1}}{(2\ell_1+1)}\,
				\dfrac{1}{r^{\ell_1+1}} - \dfrac{\tilde{R}_{\ell_1\ell_2}}{(2\ell_2+1)}\,
				\dfrac{1}{r^{\ell_2+1}}
				\right)\,,\\[2.5ex]
				\tilde{A}_{\ell_2}
				\left(
				r^{\ell_2} - \dfrac{\tilde{R}_{\ell_2\ell_1}}{(2\ell_1+1)}\,
				\dfrac{1}{r^{\ell_1+1}} - \dfrac{\tilde{R}_{\ell_2\ell_2}}{(2\ell_2+1)}\,
				\dfrac{1}{r^{\ell_2+1}}
				\right)\,.
			\end{cases}
		\end{equation}
		Comparing with the actual behavior
		\begin{equation}
			\begin{cases}
				A_{\ell_1}
				\left(
				\dfrac{k^{\ell_1}\,r^{\ell_1}}{(2\ell_1+1)!!} - \tilde{R}_{\ell_1\ell_1}\,
				\dfrac{(2\ell_1-1)!!}{k^{\ell_1+1}\,r^{\ell_1+1}} - \tilde{R}_{\ell_1\ell_2}\,
				\dfrac{(2\ell_2-1)!!}{k^{\ell_2+1}\,r^{\ell_2+1}}
				\right)\,,\\[2.5ex]
				A_{\ell_2}
				\left(
				\dfrac{k^{\ell_2}\,r^{\ell_2}}{(2\ell_2+1)!!} - \tilde{R}_{\ell_2\ell_1}\,
				\dfrac{(2\ell_1-1)!!}{k^{\ell_1+1}\,r^{\ell_1+1}} - \tilde{R}_{\ell_2\ell_2}\,
				\dfrac{(2\ell_2-1)!!}{k^{\ell_2+1}\,r^{\ell_2+1}}
				\right)\,,
			\end{cases}
		\end{equation}
		it is possible to extract
		\begin{equation}
			\begin{cases}
				\tilde{R}_{\ell_1\ell_1} =& 
				\left[(2\ell_1+1)!!\right]^2\,\dfrac{R_{\ell_1\ell_1}}{k^{2\ell_1+1}}\,,\\[2.ex]
				\tilde{R}_{\ell_1\ell_2} =& 
				(2\ell_1+1)!!\,(2\ell_2+1)!!\,\dfrac{R_{\ell_1\ell_2}}{k^{\ell_1+\ell_2+1}}\,,\\[2.ex]
				\tilde{R}_{\ell_2\ell_1} =& 
				(2\ell_1+1)!!\,(2\ell_2+1)!!\,\dfrac{R_{\ell_2\ell_1}}{k^{\ell_1+\ell_2+1}}\,,\\[2.ex]
				\tilde{R}_{\ell_2\ell_2} =& 
				\left[(2\ell_2+1)!!\right]^2\,\dfrac{R_{\ell_2\ell_2}}{k^{2\ell_2+1}}
			\end{cases}
		\end{equation}
		It is possible to use \eq{\eqref{eq:RBB}} and write
		\begin{equation}
			\begin{pmatrix}
				\tilde{R}_{\ell_1\ell_1} & 
				\tilde{R}_{\ell_1\ell_2} \\ 
				\tilde{R}_{\ell_2\ell_1} & 
				\tilde{R}_{\ell_2\ell_2} \\
			\end{pmatrix}
			=
			\begin{pmatrix}
				\left[(2\ell_1+1)!!\right]^2\left(
				\dfrac{\tan\delta_1}{k^{2\ell_1+1}}\cos^2\epsilon
				+\dfrac{\tan\delta_2}{k^{2\ell_1+1}}\sin^2\epsilon
				\right) & 
				(2\ell_1+1)!!(2\ell_2+1)!!
				\dfrac{\tan\delta_1-\tan\delta_2}{k^{\ell_1+\ell_2+1}}
				\sin\epsilon\cos\epsilon\\
				(2\ell_1+1)!!(2\ell_2+ 1)!!
				\dfrac{\tan\delta_1-\tan\delta_2}{k^{\ell_1+\ell_2+1}}
				\sin\epsilon\cos\epsilon &
				\left[(2\ell_2+1)!!\right]^2\left(
				\dfrac{\tan\delta_1}{k^{2\ell_2+1}}\sin^2\epsilon
				+\dfrac{\tan\delta_2}{k^{2\ell_2+1}}\cos^2\epsilon
				\right)
			\end{pmatrix}\,.
		\end{equation}
		Simplifying using $\ell_2=\ell+\Delta \ell$ where $\ell\equiv \ell_1$
		\begin{equation}
			\begin{pmatrix}
				\tilde{R}_{\ell_1\ell_1} & 
				\tilde{R}_{\ell_1\ell_2} \\ 
				\tilde{R}_{\ell_2\ell_1} & 
				\tilde{R}_{\ell_2\ell_2} \\
			\end{pmatrix}
			=-
			\begin{pmatrix}
				\left[(2\ell+1)!!\right]^2\left(
				a_1\,\cos^2\epsilon
				+a_2\,k^{2\Delta\ell}\sin^2\epsilon
				\right) & 
				(2\ell+1)!!(2(\ell+\Delta\ell)+1)!!
				\left(a_1-k^{2\Delta\ell}\,a_2\right)
				\dfrac{\sin2\epsilon}{2k^{\Delta\ell}}\\[2.ex]
				(2\ell+1)!!(2(\ell+\Delta\ell)+1)!!
				\left(a_1-k^{2\Delta\ell}\,a_2\right)
				\dfrac{\sin2\epsilon}{2k^{\Delta\ell}} &
				\left[(2(\ell+\Delta\ell)+1)!!\right]^2\left(
				a_1\,\dfrac{\sin^2\epsilon}{k^{2\Delta\ell}}
				+a_2\,\cos^2\epsilon
				\right)
			\end{pmatrix}\,.
		\end{equation}
		From this it is possible to infer that, in order for the $R$ matrix to not diverge
		$\epsilon\simeq k^{\Delta\ell}$ at least, this is proven in Ref.~\cite{BB}. Defining
		\begin{equation}
			e_J \equiv \frac{\epsilon}{k^{\Delta \ell}}	
		\end{equation}
		the $R$ matrix becomes
		\begin{empheq}[box=\mymath]{equation}
			\begin{pmatrix}
				\tilde{R}_{\ell_1\ell_1} & 
				\tilde{R}_{\ell_1\ell_2} \\ 
				\tilde{R}_{\ell_2\ell_1} & 
				\tilde{R}_{\ell_2\ell_2} \\
			\end{pmatrix}
			=-
			\begin{pmatrix}
				\left[(2\ell+1)!!\right]^2\,
				a_1 & 
				(2\ell+1)!!(2\ell+5)!!\,
				a_1\,
				e_J\\
				(2\ell+1)!!(2\ell+5)!!\,
				a_1\,
				e_J &
				\left[(2\ell+5)!!\right]^2\left(
				a_1\,e_J^2
				+a_2
				\right)
			\end{pmatrix}\,.			
		\end{empheq}
		\textbf{Notice }that these results are in the BB parametrization! 
		
		\subsection{BB scattering lengths $a_i$ and mixing constant $e_J$}
		In this case it is possible to solve for $a_1$, $a_2$ and $e_J$
		\begin{equation}
			\begin{cases}
				a_1 &=- \dfrac{\tilde{R}_{\ell_1\ell_1}}{\left[(2\ell+1)!!\right]^2}\,,\\[2.5ex]
				e_J &=	
				\dfrac{(2\ell+1)!!\,\tilde{R}_{\ell_1\ell_2}}
				{(2(\ell+\Delta\ell)+1)!!\,\tilde{R}_{\ell_1\ell_1}}\,,\\[2.5ex]
				a_2 &=
				\dfrac{\tilde{R}_{\ell_1\ell_2}^2-\tilde{R}_{\ell_1\ell_1}\tilde{R}_{\ell_2\ell_2}}{(2(\ell+\Delta\ell)+1)!!\,\tilde{R}_{\ell_1\ell_1}}\,,
			\end{cases}
			\mand
			\begin{cases}
				\delta_1 &\simeq -a_1 \,k^{2\ell_1+1}\,,\\
				\epsilon &\simeq e_J\,k^{\Delta \ell}\,\\
				\delta_2 & \simeq - a_2\,k^{2\ell_2+1}\,.
			\end{cases}
			\label{eq:lowEnergiesBB}
		\end{equation}
		
		
		\subsection{Stapp scattering lengths $\tilde{a}_i$ and mixing constant $\tilde{e}_J$}
		Inserting \eq{\eqref{SBB}} into \eq{\eqref{eq:epsStapp}} 
		\begin{equation}
			\sin2\tilde{\epsilon}
			=
			\frac{1}{2}\sqrt{- e^{-2 i \left(\delta _1+\delta _2\right)} \left(e^{2 i \delta _1}-e^{2 i \delta _2}\right){}^2 \sin ^2(2 \epsilon )}\,,
		\end{equation}
		where $\delta_1$, $\delta_2$ and $\epsilon$ are respectively the scattering length of the first and second channel, and the mixing angle in the BB parametrization. 
		From now on the quantities in the Stapp parametrization will have a tilde on top of them.
		Reporting here the solution from \eq{\eqref{eq:tildeepsilon}} (appendix)
		\begin{equation}
			\tilde{\epsilon}\simeq 
			- a_1\,e_J\,k^{\ell_1+\ell_2+1}
		\end{equation}
		and defining
		\begin{equation}
			\tilde{e}_J \equiv \frac{\tilde{\epsilon}}{k^{\ell_1+\ell_2+1}}
		\end{equation}
		one finds
		\begin{equation}
			\tilde{e}_J =
			-a_1\,e_J\,.
		\end{equation}
		
		In the appendix it is also proved that
		\begin{equation}
			\tilde{a}_1 = a_1 
		\end{equation}
		and
		\begin{equation}
			\tilde{a}_2 = a_2+a_1\,e_J^2\,.
		\end{equation}
		
		To summarize
		\begin{equation}
			\begin{cases}
				\tilde{a}_1 &= a_1\,,\\
				\tilde{e}_J &=
				-a_1\,e_J\,,\\
				\tilde{a}_2 &= a_2+a_1\,e_J^2\,,
			\end{cases}
			\mand
			\begin{cases}
				\tilde{\delta}_1 &\simeq - \tilde{a}_1\,k^{2\ell_1+1}\,,\\
				\tilde{\epsilon} &	\simeq
				\tilde{e}_J\,k^{\ell_1+\ell_2+1}\,,\\
				\tilde{\delta}_2 & \simeq -
				\tilde{a}_2\,k^{2\ell_2+1}\,.
			\end{cases}
		\end{equation}
		
		
		
		
	\subsection{The effective range}
	Starting from the uncoupled channel and taking \eq{\eqref{eq:connectionRs}} one can write
	\begin{equation}
		\tilde{R}_{\ell\ell}=
		\left[(2\ell+1)!!\right]^2\,
		\frac{R_{\ell\ell}}{k^{2\ell+1}}
		=
		\left[(2\ell+1)!!\right]^2\,
		\frac{1}{k^{2\ell+1}\,\cot \delta_\ell}
		\,.
	\end{equation}
	Recalling the expansion
	\begin{equation}
		k^{2\ell+1}\,\cot \delta_\ell \simeq 
		-\frac{1}{a} + \frac{1}{2}\,r_0\,k^2
		+ b\,k^4 +\mathcal{O}(k^5)\,,
	\end{equation}
	it is easy to evaluate
	\begin{equation}
		\mathcal{R}(k)=-2\left[(2\ell+1)!!\right]^2\,\frac{\tilde{R}_{\ell}(k)-\tilde{R}_{\ell\ell}(0)}{k^2\,\tilde{R}_{\ell\ell}^2(0)}\,.
	\end{equation}
	To make things simpler it is better to define $f=(2\ell+1)!!$, 
	\begin{equation}
		\mathcal{R}(k)
		=
		-\frac{2\cancel{f^2}}{k^2}\frac{\cancel{f^2}\left(\dfrac{1}{-\dfrac{1}{a} + \dfrac{1}{2}\,r_0\,k^2
		+ b\,k^4}-\dfrac{1}{-\dfrac{1}{a}}\right)}{\cancel{f^4}\,a^2}
		=
		-\frac{2}{a^2\,k^2}\,\left(
		\dfrac{1}{-\dfrac{1}{a} + \dfrac{1}{2}\,r_0\,k^2
			+ b\,k^4} +a
		\right)\,.
	\end{equation}
	This can be further simplified to
	\begin{equation}
		\mathcal{R}(k)=
		\frac{2}{a^2\,k^2}\,\left(
		\frac{2a}{-2+a\,r_0\,k^2+2ab\,k^4}
			+ a
		\right)
		=
		-\frac{2}{\cancel{a^2}\,\cancel{k^2}}\,
		\frac{\cancel{2a}-\cancel{2a}+\cancel{a^2}r_0\,\cancel{k^2}+2\cancel{a^2}b\,k^{\cancel{4}2}}{-2+a\,r_0\,k^2+2ab\,k^4}
	\end{equation}
	or
	\begin{equation}
		\mathcal{R}(k)
		=
		-2\,
		\frac{r_0+2b\,k^2}{-2+a\,r_0\,k^2+2ab\,k^4}\,.
	\end{equation}
	For $k\rightarrow 0$ 
	\begin{equation}
		\mathcal{R}(k)= r_0\left(1+ \frac{1}{2}\left(a\,r_0+\frac{4b}{r_0}\right)k^2\right) +\mathcal{O}(k^4) = r_0 +\mathcal{O}(E)\,.
	\end{equation}
	Unfortunately so far the code seems to unstable to evaluate this or there is a problem with formulas/code.
	
	
	
	
	
	
	
	
		
	\begin{appendices}
	\newcommand{\teps}{\tilde{\epsilon}}
	\newcommand{\tdone}{\tilde{\delta}_1}
	\newcommand{\tdtwo}{\tilde{\delta}_2}
	\section{Proving Stapp expansions}
		Another way to calculate $\tilde{\epsilon}$ is 
		\begin{equation}
			\sin2\teps = 
			\frac{S_{12}}{i\sqrt{\det S}}\,.
		\end{equation}	
		Using the BB version of the $S$-matrix
		\begin{equation}
			\det S = \det\left(O^T(\epsilon)
			\begin{pmatrix}
				e^{2i\delta_1} & 0 \\
				0 & e^{2i\delta_2}
			\end{pmatrix}\,
			O(\epsilon)
			\right)
			=
			\det\left(O(\epsilon)\right)^2\,
			\det\begin{pmatrix}
				e^{2i\delta_1} & 0 \\
				0 & e^{2i\delta_2}
			\end{pmatrix}
			=
			e^{2i(\delta_1+\delta_2)}\,.
		\end{equation}
		Therefore
		\begin{equation}
			\sin2\teps = 
			-i\frac{(e^{2i\delta_1}-e^{2i\delta_2})\sin\epsilon\cos\epsilon}{e^{i(\delta_1+\delta_2)}}\,.
		\end{equation}	
		Using \eq{\eqref{eq:lowEnergiesBB}}, for $k\rightarrow 0$, $\delta_1$, $\delta_2$ and $\epsilon$ get to zero and
		\begin{equation}
			\teps \simeq -\frac{1}{2}\arcsin\left(
			i\frac{(\cancel{1}+2i\delta_1-\cancel{1}-2i\delta_2)\epsilon}{1}
			\right)
			\simeq
			(\delta_1-\delta_2)\epsilon
			\simeq 
			(-a_1\,k^{2\ell+1}+a_2\,k^{2\ell+1+2\Delta\ell})e_J\,k^{\Delta\ell}\,.
		\end{equation}
		Therefore
		\begin{equation}
			\teps \simeq -a_1\,e_J\,k^{2\ell+1+\Delta\ell}
		\end{equation}
		and finally
		\begin{equation}
			\teps \simeq -a_1 \,e_J\,k^{\ell_1+\ell_1+1}\,.
			\label{eq:tildeepsilon}
		\end{equation}
		
		For $\tdone$ one can write
		\begin{equation}
			e^{2i\tdone} = \frac{S_{11}}{\cos2\teps}\,.
		\end{equation}
		Therefore
		\begin{equation}
			e^{2i\tdone} =
			\frac{
				e^{2i\delta_1}\cos^2\epsilon
				+
				e^{2i\delta_2}\sin^2\epsilon
			 }{
			 \cos\left[-
			 \cancel{2}
			 \cancel{\frac{1}{2}}\arcsin
			 \left(
			 -i\frac{(e^{2i\delta_1}-e^{2i\delta_2})\sin\epsilon\cos\epsilon}{e^{i(\delta_1+\delta_2)}}
			 \right)
			 \right]
			 }\,.
		\end{equation}
		Using 
		\begin{equation}
			\cos (\arcsin x )=
			\sqrt{1-\sin^2(\arcsin x)}=\sqrt{1-x^2}
		\end{equation}
		it simplifies to
		\begin{equation}
			e^{2i\tdone} =
			\frac{
				e^{2i\delta_1}\cos^2\epsilon
				+
				e^{2i\delta_2}\sin^2\epsilon
			}{
			\sqrt{
				1
				+
				e^{-2i(\delta_1+\delta_2)}
			(e^{2i\delta_1}-e^{2i\delta_2})^2\sin^2\epsilon\,\cos^2\epsilon}
			}\,.
		\end{equation}
		Since $\epsilon\rightarrow0$, and $\delta_i\rightarrow 0$ for small energies one finds
		\begin{equation}
			e^{2i\tdone} \simeq
				e^{2i\delta_1}
		\end{equation}
		and 
		\begin{equation}
			\tdone \simeq \delta_1 \simeq -a_1\,k^{2\ell_1+1}\,.
		\end{equation}
		Here the fact that $\delta_2 = \mathcal{O}(\delta_1)$ has been used.
		Therefore
		\begin{equation}
			\tilde{a}_1 = a_1\,.
		\end{equation}
		
		Finally one finds $\tdtwo$ using
		\begin{equation}
			e^{2i\tdtwo} = \frac{S_{22}}{\cos2\teps}\,.
		\end{equation}
		which expands to
		\begin{equation}
			e^{2i\tdtwo} =
			\frac{
				e^{2i\delta_1}\sin^2\epsilon
				+
				e^{2i\delta_2}\cos^2\epsilon
			}{
				\sqrt{
					1
					+
					e^{-2i(\delta_1+\delta_2)}
					(e^{2i\delta_1}-e^{2i\delta_2})^2\sin^2\epsilon\,\cos^2\epsilon}
			}\,.
		\end{equation}
		For small energies
		\begin{align}
			1+2i\tdtwo &\simeq
			(1+2i\delta_1)\,\epsilon^2
			+
			(1+2i\delta_2)
			\left(	
				1-\dfrac{\epsilon^2}{2}
			\right)^2
			\,,\\
			1+2i\tdtwo &\simeq
			\epsilon^2+2i\delta_1\,\epsilon^2
			+
			(1+2i\delta_2)
			\left(	
			1-\epsilon^2
			\right)
			\,,\\
			\cancel{1}+2i\tdtwo &\simeq
			\cancel{\epsilon^2}+2i\delta_1\,\epsilon^2
			+
			\cancel{1}+2i\delta_2-\cancel{\epsilon^2}
			\,,\\
			\tdtwo &\simeq \delta_2 + \delta_1\,\epsilon^2\,.
		\end{align}
		Using the expansion for small energies
		\begin{align}
			-\tilde{a}_2\,k^{2\ell_2+1}
			&=
			-a_2\,k^{2\ell_2+1}
			-
			a_1\,e_J^2\,
			k^{2\ell_1+1+2\Delta\ell}\,,\\
			\tilde{a}_2\,k^{2\ell_2+1}
			&=
			a_2\,k^{2\ell_2+1}
			+
			a_1\,e_J^2\,
			k^{2\ell_2+1}\,.
		\end{align}
		Therefore finally
		\begin{equation}
			\tilde{a}_2 = a_2+a_1\,e_J^2\,.
		\end{equation}
		
		
		
		
	\end{appendices}
	
	\begin{thebibliography}{9}
		\bibitem{advances}
		L. M. Delves, \textit{Advances in Nuclear Physics, vol.~5} (1972), Eds. M. Baranger, E. V\"ogt (Plenum Press, London, New York).
		
		\bibitem{Stapp}
		H. P. Stapp, T. Ypsilantis and N. Metropolis, Phys. Rev. \textbf{105}, 302 (1957).
		
		\bibitem{BB}
		J. M. Blatt and L. C. Biedenharn, Phys. Rev. \textbf{86}, 399 (1952).
		
		\bibitem{HS}
		M. Abramowitz, and I.A. Stegun, \textit{Handbook of Mathematical Functions with Formulas, Graphs, and Mathematical Tables}, Dover Publications Inc., New York (1965), (page 437).
		
		
	\end{thebibliography}
	
\end{document}
